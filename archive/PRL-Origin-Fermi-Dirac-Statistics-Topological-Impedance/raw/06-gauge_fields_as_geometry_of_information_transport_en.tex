\documentclass[11pt,a4paper]{article}
\usepackage[utf8]{inputenc}
\usepackage[T1]{fontenc}
\usepackage{amsmath}
\usepackage{amssymb}
\usepackage{geometry}
\usepackage{hyperref}
\usepackage{braket}
\usepackage{graphicx}
\usepackage{bm}

\geometry{left=2.5cm,right=2.5cm,top=2.5cm,bottom=2.5cm}

\title{Gauge Fields as Geometry of Information Transport: Deriving Maxwell and Yang--Mills Equations in Quantum Cellular Automata}
\author{Haobo Ma$^1$ \and Wenlin Zhang$^2$\\
\small $^1$Independent Researcher\\
\small $^2$National University of Singapore}
\date{}

\begin{document}

\mailtitle

\begin{abstract}
In standard quantum field theory, gauge fields are introduced by demanding local invariance of a Lagrangian under a prescribed Lie group, and are geometrically interpreted as connections on principal fibre bundles. In a universe described by a quantum cellular automaton (QCA), however, the microscopic ontology is a discrete network of finite-dimensional quantum systems updated by causal local unitaries. In such a discrete ontology there is no preferred global reference frame for internal degrees of freedom at different lattice sites; each cell only has access to a local Hilbert-frame.

This work develops a framework in which gauge fields arise as \textbf{translation protocols between local information frames} on a QCA. Starting from three assumptions—(i) a strictly causal, translation-invariant QCA on a regular lattice; (ii) local frame independence of physical predictions; and (iii) a minimal-distortion principle for information transport—we show:

\begin{enumerate}
\item Local redundancy in the choice of internal basis enforces the introduction of link variables as parallel-transport operators between neighbouring cells, with the standard lattice-gauge transformation law. Abelian phase redundancy yields a $U(1)$ connection; internal multi-component redundancy yields non-Abelian $G$-connections.

\item In the continuum limit of a Dirac-type QCA, the requirement that the discrete dynamics be covariant under local frame changes is equivalent to minimal coupling of a Dirac field to a gauge potential $A_\mu$, and the gauge curvature is obtained as the logarithm of elementary plaquette holonomies.

\item Imposing that the dynamics of these link variables minimize a local, gauge-invariant, quadratic information-distortion functional uniquely selects, under mild regularity and isotropy assumptions, an effective action which reduces in the continuum to the Maxwell or Yang--Mills action.

\item The microscopic gauge coupling can be expressed in terms of ratios of information-transport rates between internal and inter-site channels within the QCA, connecting running couplings to changes in effective connectivity and time-step at different probing scales.
\end{enumerate}

In this picture, fundamental interactions are not extra entities added to matter fields, but constraints on how information can be transported consistently across a discrete network of local frames. Gauge curvature measures the failure of information-parallel transport to be path-independent, and the gauge field equations arise as Euler--Lagrange equations of an information-distortion functional defined on QCA link variables.
\end{abstract}

\textbf{Keywords:} Quantum Cellular Automata; Gauge Invariance; Wilson Loops; Information Transport; Maxwell Equations; Yang--Mills Theory; Lattice Gauge Theory

\section{Introduction \& Historical Context}

From the perspective of continuous field theory, the introduction of gauge fields typically follows the "gauge principle": given a free Lagrangian with global symmetry group $G$, promote the symmetry to a spacetime point-dependent local symmetry $G(x)$, and restore invariance by introducing gauge potential $A_\mu(x)$ and covariant derivative $D_\mu = \partial_\mu - \mathrm{i} g A_\mu$. For $G = U(1)$, this yields electromagnetic fields; for $G = SU(2)\times U(1)$, $SU(3)$, etc., yields weak and strong interactions in the Standard Model. Geometrically, gauge potential is a connection one-form on a principal bundle, and curvature $F_{\mu\nu}$ corresponds to the curvature two-form of the connection.

In the non-perturbative regime, discretization of gauge theory was pioneered by Wilson's lattice gauge theory: spacetime is discretized into lattice sites and links, gauge fields are group elements $U_{x,\mu}\in G$ valued on links, curvature is given by group element products on closed Wilson loops, and in the continuum limit the Wilson action converges to Yang--Mills action. Lattice gauge theory not only provided the first controlled description of low-energy behavior in strong interactions but also supplies a natural discrete structure for quantum simulation.

On the other hand, Quantum Cellular Automata (QCA) model the physical universe as a collection of finite-dimensional quantum systems on discrete lattice sites, globally updated by local unitary operators at discrete time steps. Reviews by Arrighi, Farrelly, et al. show that QCA can unify numerous discrete spacetime quantum models, including quantum walks, lattice field theories, and partially discretized quantum field theories, and can emerge Dirac, Weyl, and Maxwell equations in appropriate continuum limits. Furthermore, Arnault et al. demonstrated that discrete-time quantum walks (DTQW) can realize $U(1)$ and non-Abelian discrete gauge theories with strict lattice gauge invariance, simulating Dirac matter coupling to electromagnetic/Yang--Mills fields in the continuum limit.

These results indicate that introducing gauge fields in discrete frameworks is not difficult; the difficulty lies in their ontological status. In traditional constructions, even on lattices, gauge fields are often viewed as "numerical tools for simulating a given continuous gauge field theory." In the discrete ontology perspective of QCA, however, lattice sites and links are the most fundamental "universe primitives," with no deeper continuous field as reference. From this perspective, natural questions arise:

\begin{itemize}
\item Can gauge fields be understood as a geometric constraint on information transport in QCA networks, rather than additional "fields"?

\item Can gauge symmetry emerge from "redundancy of local information reference frames" and "consistency of information transport," rather than as an a priori symmetry assumption?
\end{itemize}

This paper aims to provide such an interpretation from the QCA perspective. Specifically:

\begin{enumerate}
\item Viewing each cell's internal Hilbert space $\mathcal{H}_x$ as a local information reference frame, assume physical predictions are insensitive to the choice of internal basis at each point;

\item Prove that to maintain unitary and causally consistent information transport without global basis, one must necessarily introduce "parallel-transport operators" $U_{x,\mu}$ as connection fields between lattice sites, with transformation law identical to lattice gauge theory;

\item In the smooth continuum limit, combining locality, isotropy, and unitary evolution constraints, characterize the minimal-distortion action for such connection fields, proving its continuum limit uniquely selects Maxwell/Yang--Mills action;

\item Using the previously proposed principle of conservation of information rate, relate gauge coupling constant to ratios of QCA internal/external information channel rates, thereby giving microscopic information-theoretic-geometric interpretation to charge and coupling constants.
\end{enumerate}

In this framework, gauge fields are no longer "auxiliary objects artificially introduced to maintain Lagrangian invariance under a local symmetry group," but "geometric costs that must be paid to allow information to be stably transmitted and different observers to reach consistent descriptions in a discrete universe without global basis." Gauge curvature characterizes the extent to which information-parallel transport is no longer path-independent.

\section{Model \& Assumptions}

This section presents the QCA model, local reference frame redundancy and information transport principles, and defines gauge structure as the minimal geometric object satisfying these principles.

\subsection{Basic Structure of Quantum Cellular Automata}

Consider a $d$-dimensional regular lattice $\Lambda = a \mathbb{Z}^d$ with lattice spacing $a$ and time step $\Delta t$. At each lattice site $x \in \Lambda$, place a finite-dimensional Hilbert space $\mathcal{H}_{x} \cong \mathbb{C}^{N_{\mathrm{int}}}$, called internal degrees of freedom. The global Hilbert space is
$$
\mathcal{H} = \bigotimes_{x\in\Lambda} \mathcal{H}_x.
$$

One evolution step of the QCA is given by a unitary operator $G:\mathcal{H}\to\mathcal{H}$ satisfying the following conditions:

\begin{enumerate}
\item Causality: there exists finite radius $R$ such that $G$'s action on local algebra $\mathcal{A}_x$ at some lattice site $x$ depends only on degrees of freedom within radius $R$ neighborhood of $x$;

\item Translation invariance: there exists a family of translation operators $T_a$ such that $G$ commutes with $T_a$;

\item Local decomposition: $G$ can be written as finite-depth local quantum circuit $G = \prod_\ell G_\ell$, where each $G_\ell$ acts on finitely many adjacent lattice sites.
\end{enumerate}

Given internal dimensionality and locality constraints, one can construct Dirac-type QCA whose continuum limit yields Dirac equation. This paper does not depend on a specific construction, but assumes existence of such a class of "free matter QCA" whose single-particle effective Hamiltonian is
$$
H_0 = \sum_{x,\mu} \left( \psi^\dagger(x) K_\mu \psi(x + a\hat\mu) + \mathrm{h.c.} \right) + \sum_x \psi^\dagger(x) M \psi(x),
$$
where $\psi(x)\in\mathbb{C}^{N_{\mathrm{int}}}$, $K_\mu, M$ are fixed matrices, and in the continuum limit $H_0$ corresponds to some Lorentz-invariant free field (such as Dirac field).

\subsection{Local Information Reference Frame and Gauge Redundancy}

In discrete universe ontology, there is no external "God's-eye-view" global reference frame. Each cell can only access its own internal Hilbert space $\mathcal{H}_{x}$'s local basis choice. For each lattice site $x$, choose an orthonormal basis $\{\lvert e_i(x)\rangle\}_{i=1}^{N_{\mathrm{int}}}$, then state vector's coordinate representation is
$$
\psi_x = \sum_i \psi_x^i \lvert e_i(x)\rangle.
$$

Local basis choice is arbitrary: for any $V_x \in G \subset U(N_{\mathrm{int}})$, performing local transformation
$$
\lvert e_i(x)\rangle \mapsto \sum_j (V_x)_{ji} \lvert e_j(x)\rangle,
\quad
\psi_x \mapsto V_x \psi_x,
$$
physical predictions should remain unchanged. The set $\{V_x\}_{x\in\Lambda}$ constitutes local gauge group
$$
\mathcal{G} = \prod_{x\in\Lambda} G_x,
\quad G_x \cong G.
$$

This embodies "redundancy of local information reference frame."

Depending on internal structure and physical context, consider two typical cases:

\begin{enumerate}
\item $G = U(1)$: only global phase redundancy, corresponding to charge $U(1)$ gauge;

\item $G = SU(N_c)$ or more general compact Lie group: internal degrees of freedom have "color" or "flavor" multi-component structure, corresponding to non-Abelian gauge group.
\end{enumerate}

Subsequent discussion unifies using general compact Lie group $G$, with Abelian case as special case.

\subsection{Information Transport and Definition of Connection Field}

In QCA, spatial propagation of information is realized by coupling between adjacent cells. In idealized description fixing some global basis, free Hamiltonian $H_0$ contains terms
$$
\psi^\dagger(x) K_\mu \psi(x+a\hat\mu),
$$
whose physical meaning is "hopping amplitude from $x$ to $x+a\hat\mu$." However, from ontology perspective without global basis, $\psi(x)$ and $\psi(x+a\hat\mu)$ are represented in respective local bases and cannot be directly added. To describe such cross-cell information transport, a "parallel-transport operator" $U_{x,\mu}$ must be introduced between lattice sites, mapping
$$
U_{x,\mu}:\mathcal{H}_{x} \longrightarrow \mathcal{H}_{x+a\hat\mu}.
$$

\textbf{Definition 1 (Connection field/link variable).} Given gauge group $G$, a family of operators $\{U_{x,\mu}\}$ is called connection field of the QCA if for each link $(x,x+a\hat\mu)$,
$$
U_{x,\mu} \in G \subset U(\mathcal{H}_{x},\mathcal{H}_{x+a\hat\mu}}),
$$
and under local gauge transformation $\{V_x\}$ satisfies
$$
U_{x,\mu} \mapsto U'_{x,\mu} = V_{x+a\hat\mu} U_{x,\mu} V_x^\dagger.
$$

Under this definition, QCA matter Hamiltonian containing connection field naturally writes as
$$
H_{\mathrm{matter}}[U] = \sum_{x,\mu} \left( \psi^\dagger(x) K_\mu U_{x,\mu} \psi(x + a\hat\mu) + \mathrm{h.c.}\right) + \sum_x \psi^\dagger(x) M \psi(x),
$$
formally identical to "inserting group element on link" in lattice gauge theory, but here $U_{x,\mu}$ is geometric quantity necessarily introduced by local reference frame redundancy and information transport principle, not a trick "for discretizing some continuous gauge field theory."

\subsection{Information Distortion and Principle of Gauge Field Action}

Connection fields are not arbitrary degrees of freedom. Physical evolution of QCA should transport information as "gently" as possible, i.e., minimizing distortion in parallel transport while maintaining unitarity and causality. For this, introduce the following principle:

\textbf{Principle A (Local information distortion minimization).} Among all connection field dynamics compatible with given matter Hamiltonian $H_{\mathrm{matter}}[U]$ and satisfying local gauge invariance, actual physical evolution corresponds to trajectories extremizing some local gauge-invariant functional $S_{\mathrm{gauge}}[U]$.

To embody locality and isotropy, $S_{\mathrm{gauge}}[U]$ should depend only on parallel-transport operator products on minimal closed loops (plaquettes), i.e., Wilson loops
$$
W_{\Box} = \mathrm{tr} U_{\Box},
\quad
U_{\Box} = U_{x,\mu} U_{x+a\hat\mu,\nu} U_{x+a\hat\nu,\mu}^\dagger U_{x,\nu}^\dagger,
$$
and converge in continuum limit to some integrable local action density $\mathcal{L}(F_{\mu\nu})$.

Standard lattice gauge theory shows that under second-derivative and locality assumptions, the unique gauge-invariant quadratic form in continuum limit is precisely Yang--Mills action $\mathrm{tr}(F_{\mu\nu}F^{\mu\nu})$. This paper reinterprets this conclusion as "under local reference frame redundancy and information distortion minimization principles, effective action for connection field necessarily reduces to Maxwell/Yang--Mills action."

\section{Main Results (Theorems and Alignments)}

This section presents core theorems and explains their correspondence with existing theories.

\subsection{Theorem 1 (Local Basis Redundancy $\Rightarrow$ Gauge Connection Transformation Law)}

Let $G$ be a compact Lie group, $G\subset U(N_{\mathrm{int}})$ some unitary representation. Consider a given free QCA, introducing local basis transformation $V_x\in G$ at each lattice site $x$, and assume matter Hamiltonian $H_{\mathrm{matter}}[U]$ is physically equivalent under $\mathcal{G}=\prod_x G_x$. If requiring that under arbitrary local transformation $V_x$, cross-lattice transition amplitude
$$
\psi^\dagger(x) K_\mu U_{x,\mu} \psi(x+a\hat\mu)
$$
is physically invariant, then there exists and is unique a family of link operators $U_{x,\mu}\in G$ whose transformation law under $\mathcal{G}$ is
$$
U_{x,\mu} \mapsto V_{x+a\hat\mu} U_{x,\mu} V_x^\dagger.
$$
This transformation law is completely consistent with link variable transformation in lattice gauge theory.

\subsection{Theorem 2 (Minimal Coupling in Free QCA Continuum Limit)}

Let a class of QCA without connection field have single-particle effective Hamiltonian in long-wavelength limit yielding free Dirac Hamiltonian
$$
H_0 = \int \mathrm{d}^d x \, \psi^\dagger(x) \left( -\mathrm{i} \gamma^0 \gamma^i \partial_i + m \gamma^0 \right) \psi(x),
$$
satisfying translation, rotation, and discrete $\mathcal{CPT}$ symmetry. Applying Theorem 1's framework to this QCA, introducing $G$-valued connection field $U_{x,\mu}$ on links, and requiring evolution covariant under local gauge transformation. Then in smooth limit where $a,\Delta t\to 0$ and $U_{x,\mu}\to \mathbb{I}$, single-particle effective Hamiltonian of $H_{\mathrm{matter}}[U]$ is equivalent to
$$
H = \int \mathrm{d}^d x \, \psi^\dagger(x) \left( -\mathrm{i} \gamma^0 \gamma^i D_i + m \gamma^0 \right) \psi(x),
\quad
D_\mu = \partial_\mu - \mathrm{i} g A_\mu(x),
$$
where $A_\mu(x)\in \mathfrak{g}$ is Lie algebra-valued gauge potential, with relation to $U_{x,\mu}$:
$$
U_{x,\mu} = \exp\left( -\mathrm{i} g a A_\mu(x) \right) + O(a^2).
$$
In particular, for $G=U(1)$ yields charged Dirac--Maxwell coupling, for non-Abelian $G$ yields Dirac--Yang--Mills minimal coupling. This result is consistent with DTQW model conclusion that "Dirac--gauge field coupling emerges in continuum limit."

\subsection{Theorem 3 (Local Information Distortion Functional $\Rightarrow$ Maxwell / Yang--Mills Action)}

Let $S_{\mathrm{gauge}}[U]$ be action defined on link variables, satisfying:

\begin{enumerate}
\item Locality: $S_{\mathrm{gauge}}$ can be written as sum of local functions of each finite Wilson loop;

\item Gauge invariance: invariant under $U_{x,\mu}\mapsto V_{x+a\hat\mu}U_{x,\mu}V_x^\dagger$;

\item Isotropy: invariant under action of discrete rotation group on lattice;

\item Smooth limit: when $U_{x,\mu}$ values near identity, action density is positive definite quadratic form of $F_{\mu\nu}$ with no derivatives higher than second order.
\end{enumerate}

Then in continuum limit $a\to 0$, $S_{\mathrm{gauge}}[U]$ is equivalent to
$$
S_{\mathrm{gauge}} \to -\frac{1}{2}\int \mathrm{d}^{d+1}x \, \mathrm{tr}\left( F_{\mu\nu} F^{\mu\nu} \right) + O(a^2),
$$
where gauge curvature
$$
F_{\mu\nu} = \partial_\mu A_\nu - \partial_\nu A_\mu - \mathrm{i} g [A_\mu,A_\nu],
$$
reduces to Maxwell action for $G=U(1)$, yields Yang--Mills action for non-Abelian $G$. This conclusion is consistent with continuum limit of Wilson action.

\subsection{Theorem 4 (Information-Theoretic Characterization of Gauge Coupling Constant)}

In a QCA universe satisfying conservation of information rate
$$
v_{\mathrm{ext}}^2 + v_{\mathrm{int}}^2 = c^2,
$$
let each cell have $N_{\mathrm{ch}}$ channels available for external information transport (links), with coordination number $z$. Denote
$$
r_{\mathrm{int}} = \frac{\text{effective rate of internal phase update per time step}}{\text{total information rate}},
\quad
r_{\mathrm{hop}} = \frac{\text{effective rate of inter-lattice hopping per time step}}{\text{total information rate}},
$$
then under natural normalization, gauge coupling $g$ and dimensionless fine structure constant $\alpha = g^2/(4\pi)$ can be written as
$$
g^2 \propto \frac{r_{\mathrm{int}}}{z\, N_{\mathrm{ch}}\, r_{\mathrm{hop}}},
\quad
\alpha \propto \frac{r_{\mathrm{int}}}{z\, N_{\mathrm{ch}}\, r_{\mathrm{hop}}}.
$$
When probing scale changes lead to changes in effective coordination number and channel number, $\alpha$ will "run" accordingly, thereby giving discrete geometric mechanism for renormalization group flow in QCA. This result connects the previously proposed "universal conservation of information rate" with gauge coupling constant.

\section{Proofs}

This section provides proof outlines for the above theorems, with detailed derivations in appendices.

\subsection{Proof of Theorem 1: Gauge Connection Transformation Law}

Without connection field, matter Hamiltonian writes as
$$
H_0 = \sum_{x,\mu} \left( \psi^\dagger(x) K_\mu \psi(x+a\hat\mu) + \mathrm{h.c.} \right) + \sum_x \psi^\dagger(x) M \psi(x).
$$

Under local basis transformation $\psi(x)\mapsto V_x \psi(x)$, cross-lattice term becomes
$$
\psi^\dagger(x) K_\mu \psi(x+a\hat\mu)
\mapsto
\psi^\dagger(x) V_x^\dagger K_\mu V_{x+a\hat\mu} \psi(x+a\hat\mu).
$$

If $V_x$ varies with $x$, this term's matrix structure generally changes, breaking QCA's translation invariance and form of original coupling matrix $K_\mu$. To maintain physical equivalence of "matter--connection system" after local transformation, link variable $U_{x,\mu}$ must be inserted in cross-lattice term, requiring after transformation
$$
\psi^\dagger(x) K_\mu U_{x,\mu} \psi(x+a\hat\mu)
\mapsto
\psi^\dagger(x) K_\mu U'_{x,\mu} \psi(x+a\hat\mu),
$$
i.e., there exists some $U'_{x,\mu}$ making new and old Hamiltonian density forms consistent.

Writing local transformation explicitly:
$$
\psi^\dagger(x) K_\mu U_{x,\mu} \psi(x+a\hat\mu)
\mapsto
\psi^\dagger(x) V_x^\dagger K_\mu U_{x,\mu} V_{x+a\hat\mu} \psi(x+a\hat\mu).
$$

To rewrite as $\psi^\dagger(x) K_\mu U'_{x,\mu} \psi(x+a\hat\mu)$, need
$$
K_\mu U'_{x,\mu} = V_x^\dagger K_\mu U_{x,\mu} V_{x+a\hat\mu}.
$$

Assuming $K_\mu$ invertible in representation space, then
$$
U'_{x,\mu} = K_\mu^{-1} V_x^\dagger K_\mu U_{x,\mu} V_{x+a\hat\mu}.
$$

To make $U'_{x,\mu}$ still take values in $G$ and maintain translation-invariant form of $K_\mu$, natural requirement is that $K_\mu$ commutes with $G$'s representation, making $K_\mu^{-1} V_x^\dagger K_\mu = V_x^\dagger$. This can be satisfied by choosing $K_\mu$ scalar on internal space or compatible with $G$'s representation. Thus obtaining
$$
U'_{x,\mu} = V_{x+a\hat\mu} U_{x,\mu} V_x^\dagger,
$$
the standard transformation law of lattice gauge theory. Since $V_x$ arbitrary and $G$ compact group, can prove this law uniquely maintains Hamiltonian structure and local translation symmetry given $H_0$, obtaining Theorem 1.

\subsection{Proof of Theorem 2: Minimal Coupling in Continuum Limit}

Consider unified representation of Abelian case $G=U(1)$ and non-Abelian case. Let
$$
U_{x,\mu} = \exp\left( -\mathrm{i} g a A_\mu(x) \right),
\quad
A_\mu(x)\in\mathfrak{g}.
$$

In long-wavelength limit, assume $\psi(x)$ varies slowly on lattice, can write $\psi(x+a\hat\mu) = \psi(x) + a \partial_\mu \psi(x) + O(a^2)$. Cross-lattice term is
$$
\psi^\dagger(x) K_\mu U_{x,\mu} \psi(x+a\hat\mu)
= \psi^\dagger(x) K_\mu \exp\left( -\mathrm{i} g a A_\mu(x) \right) \left[ \psi(x) + a \partial_\mu \psi(x) + O(a^2) \right].
$$

Expanding to $O(a)$:
$$
\psi^\dagger(x) K_\mu \psi(x)
+ a \psi^\dagger(x) K_\mu \partial_\mu \psi(x)
- \mathrm{i} g a \psi^\dagger(x) K_\mu A_\mu(x) \psi(x) + O(a^2).
$$

For continuum limit of free QCA, existing results show there exists choice such that
$$
\sum_\mu \psi^\dagger(x) K_\mu \partial_\mu \psi(x)
\to
\psi^\dagger(x) \gamma^0 \gamma^i \partial_i \psi(x),
$$
and mass term given by $\sum_x \psi^\dagger(x)M\psi(x)$. On this basis, absorbing $-\mathrm{i} g a \psi^\dagger K_\mu A_\mu \psi$ term into derivative, can replace spatial derivative with covariant derivative
$$
\partial_\mu \mapsto D_\mu = \partial_\mu - \mathrm{i} g A_\mu(x),
$$
obtaining Dirac--gauge field minimal coupling form. Time direction coupling can be obtained through similar construction for QCA's time update operator. In actual DTQW/QCA literature, strictly lattice gauge-invariant quantum walks have been demonstrated, whose continuum limit yields Dirac field coupled Hamiltonian with external electromagnetic field; interpreting "manually inserted" connection field therein as above construction yields precise statement of Theorem 2.

\subsection{Proof of Theorem 3: Continuum Limit of Information Distortion Functional}

By Principle A, $S_{\mathrm{gauge}}[U]$ should be constructed from Wilson loops on each minimal plaquette. For a plaquette $\Box$ define
$$
U_{\Box} = U_{x,\mu} U_{x+a\hat\mu,\nu} U_{x+a\hat\nu,\mu}^\dagger U_{x,\nu}^\dagger.
$$

Requiring $S_{\mathrm{gauge}}$ invariant under $U_{x,\mu}\to V_{x+a\hat\mu}U_{x,\mu}V_x^\dagger$ means $S_{\mathrm{gauge}}$ must be constructed from $\mathrm{tr}(U_{\Box})$ and conjugate. Simplest local isotropic functional is Wilson action
$$
S_{\mathrm{W}}[U] = \sum_{\Box} \frac{\beta}{N_{\mathrm{int}}} \,\mathrm{Re}\,\mathrm{tr}\bigl(\mathbb{I} - U_{\Box}\bigr),
$$
where $\beta$ related to coupling constant $g$. For $U_{x,\mu} \approx \exp(-\mathrm{i} g a A_\mu)$ expanding in small $a$ limit, obtain
$$
U_{\Box} = \exp\left( -\mathrm{i} g a^2 F_{\mu\nu}(x) + O(a^3) \right),
$$
see Appendix A for BCH expansion calculation. Thus
$$
\mathbb{I} - U_{\Box}
= \mathrm{i} g a^2 F_{\mu\nu}(x)
+ \frac{1}{2} g^2 a^4 F_{\mu\nu}^2(x)
+ O(a^6),
$$
real part of trace to $O(a^4)$ is
$$
\mathrm{Re}\,\mathrm{tr}\bigl(\mathbb{I} - U_{\Box}\bigr)
= \frac{1}{2} g^2 a^4 \,\mathrm{tr}\bigl(F_{\mu\nu}^2(x)\bigr) + O(a^6).
$$

Approximating $\sum_{\Box}$ as $\int \mathrm{d}^{d+1}x / a^{d+1}$, obtain
$$
S_{\mathrm{W}}[U]
\to
\frac{\beta g^2}{2 N_{\mathrm{int}}} \int \mathrm{d}^{d+1}x \, \mathrm{tr}\bigl(F_{\mu\nu} F_{\mu\nu}\bigr),
$$
which is Yang--Mills action. Reduces to Maxwell action for $G=U(1)$. Conversely, can prove that given above four conditions, any other local gauge-invariant quadratic form in continuum limit only adds higher-derivative corrections to this action, thus Theorem 3 holds.

\subsection{Proof of Theorem 4: Coupling Constant and Information Rate}

In previous work, conservation of information rate principle states: for any local excitation, its external group velocity $v_{\mathrm{ext}}$ and internal state evolution velocity $v_{\mathrm{int}}$ satisfy
$$
v_{\mathrm{ext}}^2 + v_{\mathrm{int}}^2 = c^2,
$$
where $c$ is light-cone velocity of QCA. Internal state evolution velocity can be characterized by spectral width and phase rotation rate of local Hamiltonian, external group velocity given by inter-lattice transition amplitude and coordination number.

In QCA with gauge field, part of internal phase evolution is carried by "rotation" of gauge connection, corresponding from particle perspective to coupling with gauge field. Denoting
$$
r_{\mathrm{int}} = \frac{v_{\mathrm{int}}^2}{c^2},
\quad
r_{\mathrm{ext}} = \frac{v_{\mathrm{ext}}^2}{c^2},
\quad
r_{\mathrm{int}} + r_{\mathrm{ext}} = 1,
$$
can decompose $r_{\mathrm{int}}$ into gauge-related part $r_{\mathrm{gauge}}$ and "bare internal degrees of freedom" part $r_{\mathrm{bare}}$. In a QCA with $z$ nearest neighbors, $N_{\mathrm{ch}}$ transport channels per link, effective inter-link information transport capacity increases with $z N_{\mathrm{ch}}$; to maintain total rate conservation, $r_{\mathrm{gauge}}$ relative share should decrease accordingly. Natural scaling relation is
$$
g^2 \propto \frac{r_{\mathrm{gauge}}}{z\, N_{\mathrm{ch}}\, r_{\mathrm{ext}}}.
$$

In weak coupling limit, $r_{\mathrm{gauge}}$ is dominant part of $r_{\mathrm{int}}$, can approximate $r_{\mathrm{gauge}}\approx r_{\mathrm{int}}$, thus obtaining Theorem 4's relation. When probing energy scale rises, QCA's effective lattice spacing $a$ and visible coordination number $z$ change, leading to scaling changes in $g^2$ and $\alpha$, thus corresponding to renormalization group flow in continuous field theory.

\section{Model Apply}

This section gives several concrete models, demonstrating how the above general framework is implemented in specific QCA and corresponds to familiar physical theories in continuum limit.

\subsection{$U(1)$ Gauge Field in One-Dimensional Dirac QCA}

Consider one-dimensional Dirac-type QCA, whose free evolution can be written as split-step quantum walk form: at each lattice site has two-component spin $\psi(x) = (\psi_\uparrow,\psi_\downarrow)^{\mathsf T}$, time update given by alternating rotation and conditional shift. Appropriately choosing rotation angle and conditional shift, can prove continuum limit yields one-dimensional Dirac equation.

On this basis, introduce $U(1)$ phase $U_{x,\pm} = \exp[-\mathrm{i} g a A_\pm(x)]$ on each link, corresponding to left/right transport channels respectively. Requiring full-step evolution covariant under local $U(1)$ phase transformation $\psi(x)\mapsto e^{\mathrm{i}\alpha(x)}\psi(x)$, obtain
$$
U_{x,+} \mapsto e^{\mathrm{i}\alpha(x+a)} U_{x,+} e^{-\mathrm{i}\alpha(x)},
\quad
U_{x,-} \mapsto e^{\mathrm{i}\alpha(x-a)} U_{x,-} e^{-\mathrm{i}\alpha(x)}.
$$

This is precisely Abelian special case of Theorem 1. Expanding continuum limit, $U_{x,\pm} \approx \exp[-\mathrm{i} g a A_x(x)]$, yields Dirac--electromagnetic minimal coupling. Full-ensemble Wilson loops give discrete curvature of one-dimensional electric field, action adopting Wilson-type function yields Maxwell action.

\subsection{Two-Dimensional QCA and Non-Abelian $SU(2)$ Gauge Field}

On two-dimensional lattice, consider QCA with internal space $N_{\mathrm{int}} = 4$, spin and "color" each occupying two components. Let gauge group $G = SU(2)$ act on color space, internal Hamiltonian $K_\mu$ and $M$ scalar for color space. Introduce link variables
$$
U_{x,\mu} \in SU(2),
\quad
U_{x,\mu} = \exp[-\mathrm{i} g a A_\mu^a(x) T^a],
$$
where $T^a$ are $SU(2)$ generators. Local gauge transformation $\psi(x)\mapsto V_x \psi(x)$, $V_x\in SU(2)$ acting only on color space, causes link variables to transform according to Theorem 1's rule. Choosing Wilson action to establish gauge field dynamics, continuum limit yields standard equations for $SU(2)$ Yang--Mills field.

This construction echoes Arnault et al.'s work on "quantum walks and non-Abelian discrete gauge theory," which explicitly demonstrated discrete-time quantum walks with exact discrete $U(N)$ gauge invariance, yielding Yang--Mills--Dirac coupling in continuum limit.

\subsection{Electromagnetic Field from Information Geometry Perspective}

In above QCA--gauge structure, electric and magnetic fields can correspond to the following information geometric quantities:

\begin{enumerate}
\item Electric field $E_i$: phase growth rate difference on same link between adjacent time steps, characterizing desynchronization degree of local clocks. Discretely can write
   $$
   E_i(x) \sim \frac{1}{g a \Delta t} \arg\left[ U_{x,i}(t+\Delta t) U_{x,i}^\dagger(t) \right].
   $$

\item Magnetic field $B_i$: phase of Wilson loop on minimal plaquette in spatial plane, characterizing incompatibility degree of parallel transport on spatial circular path. Discretely can write
   $$
   B_i(x) \sim \frac{1}{g a^2} \arg U_{\Box}(x),
   $$
   where $\Box$ is planar plaquette perpendicular to $i$ direction.
\end{enumerate}

Aharonov--Bohm effect obtains natural interpretation in this framework: even if local electric/magnetic field strength is zero, if Wilson loop along closed path is non-trivial, information state accumulates observable global phase after parallel transport along path. This description is purely based on path-dependence of "information parallel transport," not requiring a priori continuous field background.

\section{Engineering Proposals}

This section discusses several experimental schemes for verifying above "gauge field as geometry of information transport" on realizable quantum simulation platforms.

\subsection{Electromagnetic Gauge Structure on Photonic QCA Platform}

Recent work has realized high-fidelity simulation of one-dimensional Dirac QCA on photonic platforms, implementing split-step units through programmable waveguide arrays and reconfigurable interference networks. On this basis, tunable phase shifters can be placed on optical paths between adjacent cells to realize link variables $U_{x,\pm}$. Modulating these phases at each time step can simulate time-dependent gauge potential $A_\mu(x,t)$.

Experimental steps include:

\begin{enumerate}
\item Initialize photon with specific wave packet state, evolve several steps on QCA;

\item Separately measure output interference patterns with and without link phase modulation;

\item Compare group velocity and shape changes of output wave packet center, extracting effective electric and magnetic fields;

\item By changing accumulated phase on Wilson loops, observe global shift of interference fringes, realizing discrete Aharonov--Bohm experiment.
\end{enumerate}

If experimentally measured wave packet evolution satisfies electromagnetic wave propagation laws predicted by Maxwell equations in long-wavelength limit, can view as direct support for "gauge field as parallel-transport phase" picture.

\subsection{Non-Abelian Structure in Cold Atoms and Ion Traps}

Cold atom optical lattice and ion trap systems provide natural platforms for realizing non-Abelian gauge structures, implementing internal "color" degrees of freedom through Raman coupling or multi-level structures. Existing schemes can already realize simulated lattice gauge fields on these platforms, observing characteristic features in strong coupling regime.

From QCA perspective, can design following engineering scheme:

\begin{enumerate}
\item Encode internal state of each lattice site into subspace of multi-level atom internal states, applying locally controllable $SU(2)$ or $SU(3)$ rotations to different levels as local reference frame transformations;

\item Realize parallel-transport operators $U_{x,\mu}$ between adjacent lattice sites through controlled phase gates or resonant tunneling;

\item Use time-resolved measurements to record accumulated phase and color state rotation on closed paths, thereby measuring Wilson loops;

\item Change local rotation strategies, verify physical results depend only on equivalence class $[U_{x,\mu}]$ independent of specific basis choice, thereby experimentally verifying Theorem 1's core content.
\end{enumerate}

\subsection{Experimental Estimation of Information Rate and Coupling Constant Relation}

In small-scale QCA quantum simulations, can estimate information rate and coupling constant relation through following methods:

\begin{enumerate}
\item Estimate $v_{\mathrm{ext}}$ and $v_{\mathrm{int}}$ respectively by measuring wave packet propagation velocity and local Rabi oscillation frequency;

\item By introducing controllable gauge potential, measure scattering cross-section and phase shift, back-inferring effective coupling constant $g_{\mathrm{eff}}$;

\item Change link density and available channel number $N_{\mathrm{ch}}$, observe empirical relation between $g_{\mathrm{eff}}$ and $z N_{\mathrm{ch}}$;

\item Test whether approximate $g_{\mathrm{eff}}^2 \propto r_{\mathrm{int}}/(z N_{\mathrm{ch}} r_{\mathrm{hop}})$ scaling exists, thereby providing empirical support for Theorem 4.
\end{enumerate}

These experiments need not reach high-energy physics scales to test structural predictions in QCA framework.

\section{Discussion (Risks, Boundaries, Past Work)}

\subsection{Relationship with Existing Lattice Gauge Theory}

Formally, link variables $U_{x,\mu}$, plaquette curvature $U_{\Box}$, and Wilson action obtained in this paper are completely consistent with standard lattice gauge theory. Difference lies at interpretation level:

\begin{itemize}
\item In traditional lattice field theory, lattice is auxiliary structure introduced for numerical discretization, gauge field is discrete approximation of continuous Yang--Mills field;

\item In this paper's QCA perspective, lattice sites and links are fundamental entities ontologically, gauge field is unavoidable geometric structure under local reference frame redundancy and information transport consistency.
\end{itemize}

This interpretation is consonant with some QCA literature's view that "quantum field is continuum limit of QCA," but further emphasizes "information-theoretic origin" of gauge structure.

\subsection{Comparison with Quantum Walk Simulating Gauge Fields Work}

Arnault et al.'s series of works on quantum walks and gauge fields demonstrated how to construct DTQW with exact discrete gauge invariance, yielding Maxwell and Yang--Mills equations in continuum limit. Main differences in this paper:

\begin{enumerate}
\item We start from "local information reference frame redundancy" and "information distortion minimization," not from given continuous field theory;

\item Theorems 1--3 show that within QCA ontology, gauge connection and Yang--Mills action are structures uniquely selected by general principles;

\item Introduce Theorem 4, relating coupling constant to rate allocation in QCA, providing information-theoretic-geometric interpretation for renormalization flow.
\end{enumerate}

Technically, this paper's construction can be viewed as abstraction and unification of DTQW--gauge theory.

\subsection{Risks and Boundaries}

Although above results are internally logically consistent, several boundaries warrant caution:

\begin{enumerate}
\item \textbf{Lorentz invariance emergence}: QCA generally lacks continuous Lorentz symmetry at discrete scale, whose emergence in continuum limit requires strict conditions. Existing work shows exact or approximate Lorentz symmetry achievable in one-dimensional Dirac QCA, but higher dimensions more subtle.

\item \textbf{Fermion doubling problem}: Constructing fermion fields on lattice encounters Nielsen--Ninomiya type doubling problem. QCA perspective can partially avoid this problem, but need to avoid additional "spurious species" when introducing gauge fields.

\item \textbf{Renormalization and continuum limit}: Theorem 3's continuum limit derivation relies on smoothness of $a\to 0$ and $U_{x,\mu}\to\mathbb{I}$. If QCA's underlying lattice spacing is physically indivisible Planck scale, "continuum limit" can only be approximation in effective field theory sense, whose applicability range needs specification through specific models and observation windows.

\item \textbf{Gravity gauge structure}: This paper discusses only internal gauge group, spacetime diffeomorphism and gravity's "gauge property" not yet incorporated into QCA information transport framework. How to view gravitational field as some "time scale--information volume" gauge structure requires further work combining optical metric and boundary time geometry.
\end{enumerate}

\subsection{Integration with Unified Information Rate Framework}

Previously proposed conservation of information rate and conservation of optical path framework views special relativity, mass, and gravity as results of "information rate budget" and "conservation of information volume." This work can be viewed as concretization of "gauge interaction" part in this unified framework:

\begin{itemize}
\item Gauge field existence reflects "rate allocation on internal information channels";

\item Curvature term corresponds to "information time delay difference accumulated in parallel transport";

\item Coupling constant running with energy scale corresponds to "coarse-graining effective link structure of QCA at different observation scales," thereby changing $z$ and $N_{\mathrm{ch}}$.
\end{itemize}

Thus, special relativity, gravity, and gauge interactions can all be unified in QCA as different facets of "conservation of information rate and information geometry."

\section{Conclusion}

This paper systematically constructs theory of "gauge field as geometry of information transport" in discrete ontology framework of Quantum Cellular Automata. Core conclusions include:

\begin{enumerate}
\item In discrete universe without global basis, redundancy of local information reference frame requires introducing link variables $U_{x,\mu}$ between lattice sites, whose transformation law consistent with link gauge connection in lattice gauge theory (Theorem 1).

\item Dirac-type QCA after introducing connection field yields Dirac--gauge field minimal coupling form in continuum limit under natural conditions (Theorem 2), thereby proving Maxwell and Yang--Mills equations can emerge from QCA information transport principles.

\item Requiring connection field dynamics dominated by local, gauge-invariant, isotropic and second-order "information distortion functional" uniquely selects Wilson-type action reducing to Maxwell/Yang--Mills action in continuum limit (Theorem 3).

\item Using conservation of information rate, relate gauge coupling constant to rate allocation and link structure of internal/external information channels in QCA, giving geometric-information-theoretic interpretation of coupling constant running (Theorem 4).
\end{enumerate}

From this perspective, fundamental interactions in nature are no longer "forces" added to matter, but geometric constraints on how information is allowed to be transported in discrete networks. Gauge curvature measures path-dependence of parallel transport, gauge field equations are Euler--Lagrange equations minimizing information distortion. Combining this framework with gravity, thermalization, and cosmological constant problems, expect to obtain more unified "information universe" description.

\section{Acknowledgements}

This work is theoretical research, not using numerical simulation code. Background knowledge on gauge fields and QCA references several reviews and monographs on lattice gauge theory and quantum cellular automata.

\appendix

\section{BCH Derivation from Discrete Holonomy to Continuum Field Strength}

This appendix gives standard derivation of non-Abelian field strength $F_{\mu\nu}$ corresponding to plaquette holonomy.

Consider minimal plaquette on two-dimensional lattice with vertices
$$
x,\quad x+a\hat\mu,\quad x+a\hat\mu+a\hat\nu,\quad x+a\hat\nu.
$$

Define link variables
$$
U_{x,\mu} = \exp\bigl(-\mathrm{i} g a A_\mu(x)\bigr),
$$
$$
U_{x+a\hat\mu,\nu} = \exp\bigl(-\mathrm{i} g a A_\nu(x+a\hat\mu)\bigr),
$$
$$
U_{x+a\hat\nu,\mu}^\dagger = \exp\bigl(+\mathrm{i} g a A_\mu(x+a\hat\nu)\bigr),
$$
$$
U_{x,\nu}^\dagger = \exp\bigl(+\mathrm{i} g a A_\nu(x)\bigr).
$$

Taylor expand $A_\mu(x+a\hat\nu)$, $A_\nu(x+a\hat\mu)$ to first order at $x$:
$$
A_\mu(x+a\hat\nu) = A_\mu(x) + a \partial_\nu A_\mu(x) + O(a^2),
$$
$$
A_\nu(x+a\hat\mu) = A_\nu(x) + a \partial_\mu A_\nu(x) + O(a^2).
$$

Plaquette holonomy is
$$
U_{\Box} = U_{x,\mu} U_{x+a\hat\mu,\nu} U_{x+a\hat\nu,\mu}^\dagger U_{x,\nu}^\dagger.
$$

Using Baker--Campbell--Hausdorff formula
$$
\exp(X)\exp(Y) = \exp\left(X+Y+\frac{1}{2}[X,Y] + \cdots\right),
$$
retaining to $O(a^2)$ terms. Denote
$$
X_1 = -\mathrm{i} g a A_\mu(x),
\quad
X_2 = -\mathrm{i} g a (A_\nu(x) + a \partial_\mu A_\nu(x)),
$$
$$
X_3 = +\mathrm{i} g a (A_\mu(x) + a \partial_\nu A_\mu(x)),
\quad
X_4 = +\mathrm{i} g a A_\nu(x).
$$

Then
$$
U_{\Box} = \exp(X_1)\exp(X_2)\exp(X_3)\exp(X_4).
$$

First calculate leading order of $X_1+X_2+X_3+X_4$:
$$
X_1+X_2+X_3+X_4
= -\mathrm{i} g a^2\bigl(\partial_\mu A_\nu(x) - \partial_\nu A_\mu(x)\bigr).
$$

For commutator terms, note $[X_1,X_2]$, $[X_1+X_2,X_3]$, $[X_1+X_2+X_3,X_4]$ etc. all $O(a^2)$, with dominant contribution from $[A_\mu,A_\nu]$:
$$
[X_1,X_2] = (-\mathrm{i} g a)^2 [A_\mu(x), A_\nu(x)] + O(a^3)
= - g^2 a^2 [A_\mu,A_\nu] + O(a^3).
$$

Combining all $O(a^2)$ terms, can write
$$
\ln U_{\Box}
= -\mathrm{i} g a^2\left( \partial_\mu A_\nu - \partial_\nu A_\mu - \mathrm{i} g [A_\mu,A_\nu] \right) + O(a^3)
= -\mathrm{i} g a^2 F_{\mu\nu}(x) + O(a^3),
$$
where
$$
F_{\mu\nu} = \partial_\mu A_\nu - \partial_\nu A_\mu - \mathrm{i} g [A_\mu,A_\nu].
$$

Thus
$$
U_{\Box} = \exp\bigl(-\mathrm{i} g a^2 F_{\mu\nu}(x)\bigr) + O(a^3),
$$
giving exact relation between non-Abelian field strength and plaquette holonomy. This derivation is key to Theorem 3.

\section{Topological Relation between Charge Conservation and Boundary Operators}

In QCA, $U(1)$ charge can be defined as sum of conserved operator on lattice sites, e.g.,
$$
Q = \sum_x \psi^\dagger(x)\psi(x).
$$

With gauge field present, introduce discrete electric field variable $E_{x,i}$ as conjugate momentum. Discrete Gauss law writes
$$
(\nabla\cdot E)(x) = \sum_i \left( E_{x,i} - E_{x-a\hat i,i} \right) = \rho(x),
$$
where $\rho(x)$ is charge density. Summing above relation over finite volume $V$,
$$
\sum_{x\in V} (\nabla\cdot E)(x) = \sum_{x\in V} \rho(x).
$$

Left side can be expressed as flux sum on boundary using lattice boundary operator $\partial$:
$$
\sum_{x\in V} (\nabla\cdot E)(x)
= \sum_{\text{links}\in \partial V} E_{\text{link}}.
$$

This directly embodies discrete divergence theorem, i.e., boundary operator satisfying $\partial^2 = 0$. Therefore
$$
\sum_{\text{links}\in \partial V} E_{\text{link}} = Q_{\mathrm{inside}},
$$
where $Q_{\mathrm{inside}} = \sum_{x\in V} \rho(x)$. For volume $V$ containing entire space, boundary empty, left side zero, thus obtaining total charge conservation:
$$
\frac{\mathrm{d}}{\mathrm{d}t} Q = 0.
$$

Under local gauge transformation, $E_{x,i}$ and $A_{x,i}$ transform covariantly, $\rho(x)$ definition unchanged, thus charge conservation equivalent to gauge invariance. This gives topological argument for "local gauge invariance $\Leftrightarrow$ charge conservation."

\begin{thebibliography}{99}
\bibitem{Wilson1974} K. G. Wilson, "Confinement of quarks," Phys. Rev. D 10, 2445 (1974).
\bibitem{Kogut1979} J. B. Kogut, "An introduction to lattice gauge theory and spin systems," Rev. Mod. Phys. 51, 659 (1979).
\bibitem{Seiler2025} E. Seiler, "A Gentle Introduction to Lattice Field Theory," Fortschr. Phys. 73, 2300005 (2025).
\bibitem{Arrighi2019} P. Arrighi, "An overview of Quantum Cellular Automata," Natural Computing 18, 885--899 (2019).
\bibitem{Farrelly2020} T. Farrelly, "A review of Quantum Cellular Automata," Quantum 4, 368 (2020).
\bibitem{Bisio2015} A. Bisio, G. M. D'Ariano, A. Tosini, "Quantum field as a quantum cellular automaton: The Dirac free evolution in one dimension," Ann. Phys. 354, 244--264 (2015).
\bibitem{Mallick2016} A. Mallick, S. Mandal, C. M. Chandrashekar, "Dirac Quantum Cellular Automaton from Split-step Quantum Walk," Sci. Rep. 6, 25779 (2016).
\bibitem{Arnault2016a} P. Arnault, F. Debbasch, "Quantum walks and discrete gauge theories," Phys. Rev. A 93, 052301 (2016).
\bibitem{Arnault2016b} P. Arnault, F. Debbasch, "Quantum walks and non-Abelian discrete gauge theory," Phys. Rev. A 94, 012335 (2016).
\bibitem{Mlodinow2020} L. Mlodinow, T. A. Brun, "Quantum cellular automata and quantum field theory in two and three spatial dimensions," Phys. Rev. A 102, 062222 (2020); 103, 052203 (2021).
\bibitem{Suprano2024} A. Suprano et al., "Photonic cellular automaton simulation of relativistic quantum physics," Phys. Rev. Research 6, 033136 (2024).
\bibitem{Ma2025} H. Ma et al., "Universal Conservation of Information Celerity," preprint (2025).
\bibitem{Baez1994} J. C. Baez, J. P. Muniain, \textit{Gauge Fields, Knots and Gravity}, World Scientific (1994).
\bibitem{Wen2004} X.-G. Wen, \textit{Quantum Field Theory of Many-body Systems}, Oxford University Press (2004).
\end{thebibliography}

\end{document}

