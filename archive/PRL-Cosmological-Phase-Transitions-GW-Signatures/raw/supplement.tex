\documentclass[aps,prl,onecolumn,superscriptaddress,notitlepage]{revtex4-2}
\usepackage{amsmath,amssymb,mathrsfs,braket}
\usepackage{graphicx}

\begin{document}

\title{Supplemental Material for\\
``Cosmological Phase Transitions and Gravitational Wave Signatures\\
from Topological Mass Generation in Quantum Cellular Automata''}

\maketitle

\setcounter{equation}{0}
\renewcommand{\theequation}{S\arabic{equation}}
\setcounter{figure}{0}
\renewcommand{\thefigure}{S\arabic{figure}}

%%%%%%%%%%%%%%%%%%%%%%%%%%%%%%%%%%%%%%%%%%%%%%%%%%%%%%%%%%%%%%%
\section{Split step QCA and the Dirac limit}
%%%%%%%%%%%%%%%%%%%%%%%%%%%%%%%%%%%%%%%%%%%%%%%%%%%%%%%%%%%%%%%
In this section we give the short derivation of the effective Dirac Hamiltonian
for the split step QCA used in the main text.
We consider a one dimensional split step automaton acting on a two component
spinor $\Psi_n = (\psi_{n,\uparrow},\psi_{n,\downarrow})^{\mathsf T}$ located
on a cubic lattice with spacing $\ell_{\rm cell}$.

The one step evolution operator is
\begin{equation}
  U = S_- C(\theta_2) S_+ C(\theta_1),
\end{equation}
where $C(\theta)=\exp(-i\theta\sigma_y)$ is a coin rotation on the internal
spinor space and $S_\pm$ are conditional shifts.
In momentum space we approximate the conditional shifts by a symmetric shift
acting on the chiral basis, which is accurate in the long wavelength limit,
\begin{equation}
  S_\pm(k) \simeq
  \exp\!\left(-\,\frac{i}{2}\,k \ell_{\rm cell}\,\sigma_z\right).
\end{equation}
The Floquet operator then takes the explicit form
\begin{equation}
  U(k)
  =
  e^{-i k \ell_{\rm cell}\sigma_z/2}\,
  e^{-i\theta_2\sigma_y}\,
  e^{-i k \ell_{\rm cell}\sigma_z/2}\,
  e^{-i\theta_1\sigma_y}.
  \label{eq:Uk_explicit_supp}
\end{equation}

To obtain the low energy effective Hamiltonian we expand Eq.
\eqref{eq:Uk_explicit_supp} for small quasi momentum and small coin angles.
We introduce a bookkeeping parameter $\epsilon$ and assume
$k\ell_{\rm cell} = \mathcal O(\epsilon)$ and $\theta_i = \mathcal O(\epsilon)$.
Expanding each exponential to first order and dropping
$\mathcal O(\epsilon^2)$ terms gives
\begin{align}
  e^{-i k \ell_{\rm cell}\sigma_z/2}
  &\simeq
  \mathbb{1}
  - i \frac{k\ell_{\rm cell}}{2}\sigma_z,\\[2mm]
  e^{-i\theta_i\sigma_y}
  &\simeq
  \mathbb{1}
  - i\,\theta_i\sigma_y,
\end{align}
so that
\begin{align}
  U(k)
  &\simeq
  \left(\mathbb{1}
        - i \frac{k\ell_{\rm cell}}{2}\sigma_z\right)
  \left(\mathbb{1}
        - i\theta_2\sigma_y\right)
  \left(\mathbb{1}
        - i \frac{k\ell_{\rm cell}}{2}\sigma_z\right)
  \left(\mathbb{1}
        - i\theta_1\sigma_y\right)\nonumber\\[1mm]
  &\simeq
  \mathbb{1}
  - i\bigl[(k\ell_{\rm cell})\,\sigma_z
           + (\theta_1+\theta_2)\,\sigma_y\bigr]
  + \mathcal{O}(k^2\ell_{\rm cell}^2,\theta_i^2,k\ell_{\rm cell}\theta_i).
  \label{eq:Uk_expansion_supp}
\end{align}
The terms proportional to $k\theta_i$ and $\theta_i^2$ are of higher order and
are neglected in the continuum limit.

By definition the long wavelength dynamics is generated by an effective
Hamiltonian $H_{\rm eff}(k)$ through
\begin{equation}
  U(k)
  =
  \exp\!\bigl[-\,i H_{\rm eff}(k)\Delta t/\hbar\bigr].
\end{equation}
Comparing this with Eq. \eqref{eq:Uk_expansion_supp} we identify
\begin{equation}
  H_{\rm eff}(k)
  \simeq
  \frac{\hbar}{\Delta t}
  \bigl[(k\ell_{\rm cell})\,\sigma_z
       + (\theta_1+\theta_2)\,\sigma_y\bigr].
\end{equation}
Writing the physical momentum as $p=\hbar k$, this can be cast into the Dirac
form
\begin{equation}
  H_{\rm eff}(p)
  \simeq
  c\,p\,\sigma_z + m c^2 \sigma_y,
  \label{eq:DiracHeff_supp}
\end{equation}
with emergent light speed and Dirac mass
\begin{equation}
  c = \frac{\ell_{\rm cell}}{\Delta t},
  \qquad
  m c^2 = \hbar\,\frac{\theta_1+\theta_2}{\Delta t}.
\end{equation}
In this sense the net coin angle per time step plays the role of a
topological impedance that obstructs ballistic propagation on the QCA and
appears as an effective Dirac mass in the continuum limit.
The analysis can be generalized to higher dimensional QCAs by taking tensor
products of the one dimensional construction along independent spatial
directions and performing a radial reduction, which leads to the same leading
order Dirac structure for radial modes.

%%%%%%%%%%%%%%%%%%%%%%%%%%%%%%%%%%%%%%%%%%%%%%%%%%%%%%%%%%%%%%%
\section{Holographic entropy bound and holographic redundancy}
%%%%%%%%%%%%%%%%%%%%%%%%%%%%%%%%%%%%%%%%%%%%%%%%%%%%%%%%%%%%%%%
Here we collect the simple geometric relations that underlie the holographic
redundancy parameter $\xi$ and its connection to the cosmological constant
$\Lambda$.
We consider a de Sitter patch of radius $R_\Lambda$.
In Planck units, the Gibbons Hawking entropy of the de Sitter horizon is
\begin{equation}
  S_\Lambda
  =
  \frac{A_\Lambda}{4}
  =
  \pi R_\Lambda^2,
  \label{eq:dS_entropy}
\end{equation}
where $A_\Lambda = 4\pi R_\Lambda^2$ is the horizon area.
For a pure de Sitter universe with cosmological constant $\Lambda$ we have
\begin{equation}
  R_\Lambda^2
  \sim
  \frac{3}{\Lambda},
\end{equation}
and consequently $S_\Lambda \propto \Lambda^{-1}$.

In the QCA discretization we tile the three dimensional spatial volume of the
patch with cubic cells of linear size $\ell_{\rm cell}$.
The number of bulk cells and surface pixels are estimated as
\begin{equation}
  N_{\rm bulk}
  \simeq
  \frac{4\pi R_\Lambda^3}{3\ell_{\rm cell}^3},
  \qquad
  N_{\rm surf}
  \simeq
  \frac{4\pi R_\Lambda^2}{\ell_{\rm cell}^2}.
\end{equation}
The holographic redundancy parameter
\begin{equation}
  \xi
  \equiv
  \frac{N_{\rm bulk}}{N_{\rm surf}},
\end{equation}
counts how many bulk QCA cells correspond, on average, to one independent
horizon degree of freedom.
With the above expressions we obtain
\begin{equation}
  \xi
  =
  \frac{N_{\rm bulk}}{N_{\rm surf}}
  =
  \frac{R_\Lambda}{3\,\ell_{\rm cell}},
  \label{eq:xi_def_supp}
\end{equation}
up to corrections of order unity that arise from the precise choice of patch
shape and tiling near the boundary.

Solving Eq. \eqref{eq:xi_def_supp} for the lattice spacing gives
\begin{equation}
  \ell_{\rm cell}
  =
  \frac{R_\Lambda}{3\xi}.
  \label{eq:ell_from_xi_supp}
\end{equation}
Combining Eq. \eqref{eq:ell_from_xi_supp} with the de Sitter relation
$R_\Lambda^2 \simeq 3/\Lambda$ leads to
\begin{equation}
  \Lambda \ell_{\rm cell}^2
  \simeq
  \frac{1}{3}\,\frac{1}{\xi^2},
\end{equation}
for the simplest geometric choice.
This shows that the product $\Lambda \ell_{\rm cell}^2$ is parametrically
suppressed by the square of the redundancy factor.
More careful accounting of geometric factors, the precise relation between the
horizon radius and the cosmological constant, and the discretization of the
horizon surface into QCA pixels modifies the numerical coefficient by a factor
of order unity.
For the conventions adopted in the main text, these refinements yield
\begin{equation}
  \Lambda \ell_{\rm cell}^2
  \approx \frac{9}{64}\,\frac{1}{\xi^2},
  \label{eq:Lambda_xi_refined_supp}
\end{equation}
which we quote as Eq. (9) in the Letter.
The precise numerical factor is not important for our purposes, since it does
not affect the parametric scaling with $\Lambda$ and $\xi$, nor the location
of the gravitational wave cutoff on a logarithmic frequency scale.

%%%%%%%%%%%%%%%%%%%%%%%%%%%%%%%%%%%%%%%%%%%%%%%%%%%%%%%%%%%%%%%
\section{Lattice dispersion and gravitational wave window function}
%%%%%%%%%%%%%%%%%%%%%%%%%%%%%%%%%%%%%%%%%%%%%%%%%%%%%%%%%%%%%%%
We now derive the lattice dispersion relation and discuss the origin of the
window function $W(f)$ that multiplies the continuum template
$\Omega_{\rm cont}(f)$.

\subsection{Discrete wave equation and Nyquist cutoff}
We model the propagation of a transverse traceless metric perturbation
$h_{ij}$ on the QCA as a linear wave on a cubic lattice.
For simplicity we suppress tensor indices and focus on a single scalar mode
$h_n(t)$ living on lattice sites $n$ with spacing $\ell_{\rm cell}$, obeying
\begin{equation}
  \ddot{h}_n
  =
  \frac{c^2}{\ell_{\rm cell}^2}
  \bigl(h_{n+1} - 2 h_n + h_{n-1}\bigr)
  + s_n(t),
  \label{eq:discrete_wave_supp}
\end{equation}
where $s_n(t)$ is a source term built from the post transition impedance
field and $c$ is the emergent light speed.
In the absence of sources we seek plane wave solutions of the form
\begin{equation}
  h_n(t)
  =
  \Re\!\left\{
    \tilde{h}(k)\,
    e^{i(k n\ell_{\rm cell} - \omega t)}
  \right\}.
\end{equation}
Substituting this into Eq. \eqref{eq:discrete_wave_supp} with $s_n(t)=0$
gives
\begin{equation}
  -\omega^2 \tilde{h}(k)
  =
  \frac{c^2}{\ell_{\rm cell}^2}
  \bigl(e^{ik\ell_{\rm cell}}
       -2
       +e^{-ik\ell_{\rm cell}}\bigr)\tilde{h}(k),
\end{equation}
which simplifies to
\begin{equation}
  \omega^2(k)
  =
  \frac{4c^2}{\ell_{\rm cell}^2}\,
  \sin^2\!\left(\frac{k\ell_{\rm cell}}{2}\right).
  \label{eq:dispersion_supp}
\end{equation}
The allowed wave numbers lie in the first Brillouin zone
$k\in[-\pi/\ell_{\rm cell},\pi/\ell_{\rm cell}]$, so the maximum frequency
is
\begin{equation}
  \omega_{\rm max}
  =
  \frac{2c}{\ell_{\rm cell}},
\end{equation}
attained at $|k|=\pi/\ell_{\rm cell}$.
Expressed as a frequency this Nyquist cutoff reads
\begin{equation}
  f_{\rm cut}
  =
  \frac{\omega_{\rm max}}{2\pi}
  \simeq
  \frac{c}{\pi\ell_{\rm cell}},
  \label{eq:fcut_supp}
\end{equation}
which is Eq. (11) of the Letter.
This expression shows explicitly that the ultraviolet end of the lattice
spectrum is fixed by the QCA cell size.

\subsection{Toy numerical implementation and window function}
To connect the discrete dynamics to the continuum template for the stochastic
gravitational wave background we implement a simple toy model.
We evolve Eq. \eqref{eq:discrete_wave_supp} on a periodic lattice with
$N$ sites using a leapfrog integrator with time step $\Delta t$ chosen such
that $c\Delta t/\ell_{\rm cell}<1$ for stability.
The source $s_n(t)$ is taken to be a superposition of short pulses localized
in both space and time during a finite interval around the transition time
$t_\ast$.
The pulses have random phases and amplitudes drawn from a fixed distribution,
which mimics the stochastic structure of the post transition impedance field
and the associated anisotropic stress.
For each realization we record $h_n(t)$ over an interval much longer than the
duration of the source, then compute the discrete Fourier transform in time
and space,
\begin{equation}
  \tilde{h}(k,\omega)
  =
  \sum_{n=0}^{N-1}
  \int dt\,
  h_n(t)\,
  e^{-i(k n\ell_{\rm cell} - \omega t)}.
\end{equation}
The energy density in gravitational waves per logarithmic frequency interval
is obtained from $|\tilde{h}(k,\omega)|^2$ in the usual way.
Averaging over realizations and integrating over the Brillouin zone yields a
numerical estimate of $\Omega_{\rm GW}(f)$.

We find that the resulting spectra are well described, over a wide range of
toy model parameters, by a product form
\begin{equation}
  \Omega_{\rm QCA}(f)
  =
  \Omega_{\rm cont}(f)\,W(f),
\end{equation}
where $\Omega_{\rm cont}(f)$ is the continuum template for acoustic
turbulence in a first order transition and $W(f)$ is a universal window
function that depends only on $f/f_{\rm cut}$.
Empirically $W(f)$ is accurately fitted by a squared sinc profile,
\begin{equation}
  W(f)
  =
  \left[
    \frac{\sin(\pi f/f_{\rm cut})}{\pi f/f_{\rm cut}}
  \right]^2,
  \label{eq:lattice_window_supp}
\end{equation}
with $f_{\rm cut}$ given by Eq. \eqref{eq:fcut_supp}.
The functional form \eqref{eq:lattice_window_supp} can be understood as the
square of the discrete Fourier transform of a compact source sampled on a
uniform lattice, analogous to the familiar window functions in signal
processing.
For $f\ll f_{\rm cut}$, Eq. \eqref{eq:lattice_window_supp} approaches unity
and the QCA spectrum coincides with the continuum prediction.
As $f$ approaches $f_{\rm cut}$ the window induces a sharp suppression and
oscillatory structure that can be interpreted as Bragg like features
associated with the underlying lattice.
Modes with $f>f_{\rm cut}$ do not propagate on the QCA, leading to a hard
ultraviolet cutoff in the GW spectrum.

\end{document}


