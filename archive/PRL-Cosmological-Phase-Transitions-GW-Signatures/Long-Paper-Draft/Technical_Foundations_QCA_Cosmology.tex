\documentclass[aps,prd,notitlepage,10pt,superscriptaddress,nofootinbib,longbibliography]{revtex4-2}
\usepackage{amsmath,amssymb,amsthm,mathrsfs,braket,graphicx,hyperref,xcolor}

\hypersetup{colorlinks=true,linkcolor=blue,citecolor=blue,urlcolor=blue}

\begin{document}

\title{Technical Foundations of Quantum Cellular Automaton Cosmology:\\
Topological Impedance, Holographic Redundancy, and Lattice Gravitational Waves}

\author{Haobo Ma}
\affiliation{Independent Researcher}
\author{Wenlin Zhang}
\affiliation{National University of Singapore}

\date{\today}

\begin{abstract}
This paper provides the detailed mathematical derivation and physical justification for the cosmological framework proposed in our companion Letter. We construct a concrete 3D Quantum Cellular Automaton (QCA) model based on split-step quantum walks, demonstrating how the Dirac equation and mass term emerge from the continuum limit of local coin parameters. We explicitly derive the topological winding number associated with the QCA Floquet operator, identifying mass as a topological impedance generated by chiral symmetry breaking. We then detail the ``Micro-Parallelism'' geometric construction that yields the Standard Model gauge group $SU(3)\times SU(2)\times U(1)$ from the minimal unitary symmetries of the local Hilbert space. Furthermore, we provide a rigorous derivation of the holographic scaling relation $\Lambda \ell_{\rm cell}^2 \propto \xi^{-2}$, interpreting the smallness of the cosmological constant as a consequence of immense holographic redundancy $\xi$ in the discrete bulk. Finally, we compute the full dispersion relation for linear gravitational perturbations on the QCA lattice, obtaining the specific transfer function for the stochastic gravitational wave background that features a distinctive UV cutoff and Bragg peak, serving as a smoking-gun signature for discrete spacetime.
\end{abstract}

\maketitle

\tableofcontents

\section{Introduction}
The purpose of this technical companion is to substantiate the claims made in Ref.~\cite{PRL_Letter} regarding the QCA origin of mass, gauge symmetries, and the cosmological constant. While the Letter focuses on the phenomenological consequences, here we present the underlying microscopic definitions and derivations.

\section{QCA Micro-Dynamics and Topological Mass}

\subsection{The Split-Step Quantum Walk Model}
We consider a one dimensional split-step QCA on a lattice with spacing
$\ell_{\rm cell}$ and sites $n\in\mathbb{Z}$. At each site the local Hilbert space
is a two-component spinor
\begin{equation}
    \Psi_n(t) =
    \begin{pmatrix}
        \psi_{n,L}(t)\\
        \psi_{n,R}(t)
    \end{pmatrix},
\end{equation}
representing left and right chiral components. One discrete time step
$\Delta t$ is given by
\begin{equation}
    \Psi(t+\Delta t) = U\,\Psi(t), \qquad
    U = S_-\,C(\theta_2)\,S_+\,C(\theta_1),
    \label{eq:supp_splitstep_def}
\end{equation}
where the local coin rotations about the $y$ axis are
\begin{equation}
    C(\theta_j) = e^{-i\theta_j\sigma_y}
    =
    \begin{pmatrix}
        \cos\theta_j & -\sin\theta_j\\
        \sin\theta_j & \cos\theta_j
    \end{pmatrix},
    \qquad j=1,2,
\end{equation}
and the conditional shifts act as
\begin{align}
    (S_+\Psi)_n &=
    \begin{pmatrix}
        \psi_{n+1,L}\\
        \psi_{n,R}
    \end{pmatrix},
    &
    (S_-\Psi)_n &=
    \begin{pmatrix}
        \psi_{n,L}\\
        \psi_{n-1,R}
    \end{pmatrix}.
\end{align}
Going to momentum space via
\begin{equation}
    \Psi_n(t)
    = \int_{-\pi/\ell_{\rm cell}}^{\pi/\ell_{\rm cell}}
      \frac{\mathrm{d}k}{2\pi}\,
      e^{ikn\ell_{\rm cell}}\Psi(k,t),
\end{equation}
the shifts become
\begin{equation}
    S_+(k) = e^{+ik\ell_{\rm cell}P_L},\qquad
    S_-(k) = e^{-ik\ell_{\rm cell}P_R},
\end{equation}
with chiral projectors
$P_{L,R} = (\mathbb{1}\mp\sigma_z)/2$. The one step Floquet operator is
\begin{equation}
    \tilde U(k) = S_-(k)\,C(\theta_2)\,S_+(k)\,C(\theta_1).
\end{equation}
A straightforward matrix multiplication yields
\begin{equation}
    \tilde U(k) =
    \begin{pmatrix}
        -\sin\theta_1\sin\theta_2 + e^{-ik\ell_{\rm cell}}\cos\theta_1\cos\theta_2
        &
        -\sin\theta_2\cos\theta_1 - e^{-ik\ell_{\rm cell}}\sin\theta_1\cos\theta_2
        \\[4pt]
        e^{+ik\ell_{\rm cell}}\sin\theta_1\cos\theta_2 + \sin\theta_2\cos\theta_1
        &
        e^{+ik\ell_{\rm cell}}\cos\theta_1\cos\theta_2 - \sin\theta_1\sin\theta_2
    \end{pmatrix}.
    \label{eq:supp_Uk_full}
\end{equation}
Since $\tilde U(k)\in SU(2)$, its eigenvalues can be written as
$e^{-i\varepsilon(k)\Delta t}$ with a quasi-energy $\varepsilon(k)$ satisfying
\begin{equation}
    \cos\big(\varepsilon(k)\Delta t\big)
    = \tfrac{1}{2}\,\mathrm{Tr}\,\tilde U(k)
    = \cos(k\ell_{\rm cell})\cos\theta_1\cos\theta_2
      - \sin\theta_1\sin\theta_2.
    \label{eq:supp_cos_eps}
\end{equation}
Equation \eqref{eq:supp_cos_eps} makes explicit the dispersion relation
of the split-step walk.

\subsection{Continuum limit and effective Dirac Hamiltonian}

We now take the continuum limit in which both $k\ell_{\rm cell}$ and the coin
angles $|\theta_{1,2}|$ are small. Expanding Eq.~\eqref{eq:supp_cos_eps}
to leading order in these small quantities we find
\begin{align}
   \cos\big(\varepsilon\Delta t\big)
   &\simeq
   \Big(1 - \tfrac{1}{2}k^2\ell_{\rm cell}^2\Big)
   \Big(1 - \tfrac{1}{2}(\theta_1^2+\theta_2^2)\Big)
   - \theta_1\theta_2
   \nonumber\\
   &\simeq 1 - \frac{k^2\ell_{\rm cell}^2}{2} - \frac{(\theta_1+\theta_2)^2}{2}
   + \mathcal{O}(k^2\theta^2,\theta^4).
\end{align}
For small quasi-energy $\varepsilon\Delta t\ll1$, we also have
\begin{equation}
    \cos(\varepsilon\Delta t)
    \simeq 1 - \frac{(\varepsilon\Delta t)^2}{2}.
\end{equation}
Equating the two expansions gives
\begin{equation}
    \varepsilon^2(k)\Delta t^2
    \simeq k^2\ell_{\rm cell}^2 + (\theta_1+\theta_2)^2.
    \label{eq:supp_dispersion}
\end{equation}

Introducing the emergent light speed
\begin{equation}
    c \equiv \frac{\ell_{\rm cell}}{\Delta t},
\end{equation}
and defining the effective mass via
\begin{equation}
    m c^2 \equiv \frac{\hbar}{\Delta t}(\theta_1+\theta_2),
\end{equation}
Eq.~\eqref{eq:supp_dispersion} becomes
\begin{equation}
    \varepsilon^2(k)
    \simeq c^2k^2 + m^2c^4,
\end{equation}
which is the standard relativistic energy-momentum relation.

More directly, we can expand the Floquet operator itself. Using the
Baker-Campbell-Hausdorff formula to first order in $k\ell_{\rm cell}$
and $\theta_{1,2}$, one finds
\begin{equation}
    \tilde U(k) \simeq
    \exp\left[
        -i\left(
            k\ell_{\rm cell}\,\sigma_z
            + (\theta_1+\theta_2)\sigma_y
        \right)
    \right]
    + \mathcal{O}(k^2,\theta^2,k\theta).
\end{equation}
Comparing this with the general form
$\tilde U(k)\simeq\mathbb{1}-iH_{\rm eff}(k)\Delta t/\hbar$, we obtain
the effective Hamiltonian
\begin{equation}
    H_{\rm eff}(k)
    \simeq c p\,\sigma_z + m c^2\sigma_y,
    \qquad p\equiv\hbar k,
\end{equation}
which is precisely the Dirac Hamiltonian in one spatial dimension.
This establishes the claimed identification between the QCA coin
parameters $(\theta_1,\theta_2)$ and the Dirac mass $m$ quoted in the main text.

\subsection{Chiral symmetry and winding number}

We now focus on the single-coin case $\theta_2=0$, which already captures
the essential topology. Equation~\eqref{eq:supp_Uk_full} reduces to
\begin{equation}
   \tilde U(k;\theta) =
   \begin{pmatrix}
      e^{-ik\ell_{\rm cell}}\cos\theta &
      -e^{-ik\ell_{\rm cell}}\sin\theta\\[3pt]
      e^{+ik\ell_{\rm cell}}\sin\theta &
      e^{+ik\ell_{\rm cell}}\cos\theta
   \end{pmatrix},
   \qquad \theta\equiv\theta_1.
   \label{eq:supp_Uk_theta}
\end{equation}
This walk has a chiral symmetry generated by
\begin{equation}
   \Gamma = \sigma_x,\qquad
   \Gamma^2=\mathbb{1},\qquad
   \Gamma\,\tilde U(k;\theta)\,\Gamma = \tilde U^\dagger(k;\theta),
\end{equation}
placing it in the AIII/BDI class. In the eigenbasis of $\Gamma$,
spanned by
\begin{equation}
   |+\rangle = \frac{|L\rangle+|R\rangle}{\sqrt{2}},
   \qquad
   |-\rangle = \frac{|L\rangle-|R\rangle}{\sqrt{2}},
\end{equation}
the chiral operator is diagonal, $\Gamma'=\sigma_z$, and the Floquet
operator takes the off-diagonal form
\begin{equation}
   \tilde U'(k;\theta)
   = T\,\tilde U(k;\theta)\,T^\dagger
   =
   \begin{pmatrix}
      0 & q(k)\\
      q^\dagger(k) & 0
   \end{pmatrix},
\end{equation}
where
\begin{equation}
   T = \frac{1}{\sqrt{2}}
   \begin{pmatrix}
      1 & 1\\
      1 & -1
   \end{pmatrix},
\end{equation}
and a short computation gives
\begin{equation}
    q(k) = -\sin\theta\,e^{-ik\ell_{\rm cell}}.
\end{equation}
For a one dimensional chiral unitary of this form the topological
invariant is a winding number
\begin{equation}
   \mathcal{W}
   = \frac{1}{2\pi i}
     \int_{-\pi/\ell_{\rm cell}}^{\pi/\ell_{\rm cell}}
     \mathrm{d}k\,\frac{\mathrm{d}}{\mathrm{d}k}\log\det q(k)
   = \frac{1}{2\pi}
     \int_{-\pi/\ell_{\rm cell}}^{\pi/\ell_{\rm cell}}
     \mathrm{d}k\,\partial_k\arg q(k).
    \label{eq:supp_winding_def}
\end{equation}
Since $\det q(k)=q(k)$ is a nonzero complex number for $\theta\neq0$,
\begin{equation}
    \arg q(k) = -k\ell_{\rm cell} + \mathrm{const},
\end{equation}
and therefore
\begin{equation}
   \partial_k\arg q(k) = -\ell_{\rm cell},
\end{equation}
which is independent of $k$. Inserting this into
Eq.~\eqref{eq:supp_winding_def} yields
\begin{equation}
   \mathcal{W} =
   -\frac{1}{2\pi}
   \int_{-\pi/\ell_{\rm cell}}^{\pi/\ell_{\rm cell}}
   \ell_{\rm cell}\,\mathrm{d}k
   = -1,
\end{equation}
for any $0<|\theta|<\pi$. At the special point $\theta=0$ the walk
reduces to a pure shift, $q(k)=0$ and the winding number is ill defined.
Approaching $\theta\to0^\pm$ from either side shows that the nontrivial
Phase has $|\mathcal{W}|=1$, while the point $\theta=0$ is a topologically
trivial gap closing.

Because the effective Dirac mass $m$ is proportional to $\theta$, the sign
and magnitude of $m$ label distinct topological sectors. The cosmological
phase transition in which $\theta$ freezes from $0$ to a nonzero value
therefore realizes a topological phase transition between $\mathcal{W}=0$
and $|\mathcal{W}|=1$, and we interpret $m$ as a topological impedance
parameter obstructing ballistic information flow.

\section{Information scrambling and effective free energy}
\label{sec:free_energy}

\subsection{Information temperature and scrambling time}

We consider a comoving Hubble patch containing $N$ QCA cells.
During each time step $\Delta t$ the global update operator $U(t)$
is composed of a layer of local gates acting on neighboring cells.
We define an information temperature $T$ via the average number of
nontrivial local gates per cell per unit time, or equivalently via the
Lyapunov-like scrambling rate extracted from an out-of-time-ordered
correlator (OTOC)
\begin{equation}
    C(t) \equiv
    \frac{1}{d}\mathrm{Tr}\Big(
        [W(t),V]^\dagger[W(t),V]
    \Big),
    \qquad
    W(t)=U^\dagger(t)WU(t),
\end{equation}
where $V$ and $W$ are local operators and $d$ is the Hilbert space
dimension of the patch. In a chaotic QCA one typically finds
\begin{equation}
    C(t) \sim e^{2\lambda_s t},
\end{equation}
defining the scrambling rate $\lambda_s$. We take $T$ as a monotonic
function of $\lambda_s$, and for simplicity assume a linear relation
\begin{equation}
    \lambda_s(T) = \gamma T,
\end{equation}
with some positive constant $\gamma$. The scrambling time is then
\begin{equation}
    \tau_s(T) \equiv \lambda_s^{-1}(T) = \gamma^{-1} T^{-1}.
\end{equation}

During the early universe at high $T$ we have
$\tau_s(T)\ll H^{-1}(T)$, where $H$ is the Hubble rate. The QCA
can explore the space of local update rules and coin angles many times
within a Hubble time, so the coarse-grained coin angle
$\theta$ equilibrates to $\langle\theta\rangle=0$. As the universe
expands and $T$ decreases, there is a critical temperature $T_c$
defined implicitly by
\begin{equation}
   \tau_s(T_c) \simeq H^{-1}(T_c).
\end{equation}
For $T<T_c$ the QCA falls out of equilibrium: the local update rule
can no longer decorrelate within one Hubble time and the coarse-grained
coin angle $\theta$ effectively freezes to a nonzero value on each
causal patch. This is the microscopic origin of the ``impedance
freezing'' transition described in the main text.

\subsection{Free energy from energy cost and combinatorial entropy}

Let $\theta$ denote the spatial average of the coin angle over a Hubble
patch,
\begin{equation}
    \theta = \frac{1}{N}\sum_{x\in{\rm patch}}\theta_x.
\end{equation}
We define an energy density $\varepsilon(\theta)$ as the effective
cost in terms of reduced propagation speed and increased impedance
when the patch has an average coin angle $\theta$. For small $\theta$
symmetry under $\theta\to-\theta$ implies an even expansion,
\begin{equation}
    \varepsilon(\theta)
    = a_0\theta^2 + a_4\theta^4 + \mathcal{O}(\theta^6),
    \qquad a_0>0.
\end{equation}
On the other hand, let $\mathcal{N}(\theta)$ be the number of microscopic
QCA configurations (sequences of local gates) compatible with a given
macroscopic value of $\theta$. The corresponding information entropy
density is
\begin{equation}
    s(\theta) = \frac{1}{V_{\rm patch}}\log\mathcal{N}(\theta),
\end{equation}
where $V_{\rm patch}$ is the physical volume of the Hubble patch.
Again by symmetry we expand as
\begin{equation}
    s(\theta) = s_0 + s_2\theta^2 - s_4\theta^4
    + \mathcal{O}(\theta^6),
    \qquad s_2>0.
\end{equation}
The sign $s_2>0$ reflects the fact that a small nonzero $\theta$ opens
up additional topological sectors in configuration space, thus increasing
the number of allowed microhistories.

The information-theoretic free energy density is defined as
\begin{equation}
    \mathcal{F}(\theta,T)
    = \varepsilon(\theta) - T s(\theta).
\end{equation}
Substituting the expansions yields a Landau-type potential
\begin{equation}
    \mathcal{F}(\theta,T)
    = \alpha(T)\theta^2 + \beta(T)\theta^4
      + \mathcal{O}(\theta^6),
\end{equation}
with coefficients
\begin{equation}
    \alpha(T) = a_0 - T s_2, \qquad
    \beta(T) = a_4 + T s_4.
\end{equation}
The critical temperature at which the curvature at the origin changes
sign is
\begin{equation}
    T_c = \frac{a_0}{s_2}.
\end{equation}
For $T>T_c$ we have $\alpha(T)>0$ and the free energy has a single
minimum at $\theta_0(T)=0$, corresponding to a chirally symmetric
massless phase. For $T<T_c$ the quadratic coefficient becomes negative,
$\alpha(T)<0$, and the minimum splits into two degenerate minima at
\begin{equation}
    \theta_0^2(T)
    = -\frac{\alpha(T)}{2\beta(T)}
    = \frac{T s_2 - a_0}{2(a_4+T s_4)}.
\end{equation}
On each causal patch the QCA spontaneously selects one of $\pm\theta_0$.
Because the sign and magnitude of $\theta_0$ determine the sign and size
of the Dirac mass $m\propto\theta_0$, this is a topological
symmetry-breaking phase transition that dynamically generates mass
from information.

Combining this picture with the Kibble-Zurek condition
\begin{equation}
    \tau_s(T_c)\sim H^{-1}(T_c),
\end{equation}
we expect the domains of fixed $\theta_0$ (and thus fixed topological
winding $\mathcal{W}$) to form with a characteristic size set by
the freeze-out scale, and a corresponding network of topological
defects whose collisions source the gravitational waves discussed in
the main text.

\section{Gauge Groups from Micro-Parallelism}

\subsection{Local Hilbert Space Factorization}
We postulate the ``Micro-Parallelism Axiom'':
\begin{equation}
    \mathcal{H}_{\rm loc} \cong \mathcal{H}_{\rm matter} \otimes \mathbb{C}^3_{\rm space} \otimes \mathbb{C}^2_{\rm time} \otimes U(1)_{\rm phase}.
\end{equation}
Here, $\mathbb{C}^3_{\rm space}$ represents three orthogonal micro-channels for spatial propagation, and $\mathbb{C}^2_{\rm time}$ represents the input/output (or past/future) time-bin structure of the update rule.

\subsection{Emergence of $SU(3) \times SU(2) \times U(1)$}
We demand that the local symmetry group $G_{\rm loc}$ be the maximal group of unitary transformations that:
1. Preserve the local inner product (Unitarity).
2. Act transitively on each factor space (Irreducibility).
3. Are connected to the identity (Continuous symmetry).
4. Introduce no accidental degeneracies (Minimality).

Under these axioms:
\begin{itemize}
    \item On $\mathbb{C}^3_{\rm space}$, the minimal group is $SU(3)$.
    \item On $\mathbb{C}^2_{\rm time}$, the minimal group is $SU(2)$.
    \item On $U(1)_{\rm phase}$, the group is $U(1)$.
\end{itemize}
The kinematical symmetry group is thus $SU(3) \times SU(2) \times U(1)$. Gauging these symmetries involves identifying elements that act identically on physical states. The centers $Z(SU(3)) \cong \mathbb{Z}_3$ and $Z(SU(2)) \cong \mathbb{Z}_2$ can be compensated by global phase shifts. Quotienting by this common center yields the physical gauge group:
\begin{equation}
    G_{\Sigma} \cong \frac{SU(3) \times SU(2) \times U(1)}{\mathbb{Z}_6},
\end{equation}
reproducing the Standard Model structure from pure information geometry.

\section{Holographic redundancy and the relation
$\Lambda\ell_{\rm cell}^2\propto 1/\xi^2$}
\label{sec:xi_Lambda}

\subsection{Bulk and surface cell counting}

We consider a de Sitter universe with cosmological constant $\Lambda$.
The horizon radius is
\begin{equation}
    R_\Lambda = \sqrt{\frac{3}{\Lambda}},
\end{equation}
and the horizon area is $A=4\pi R_\Lambda^2$. If we tile the bulk with
QCA cells of linear size $\ell_{\rm cell}$, the number of bulk cells
inside the horizon is approximately
\begin{equation}
    N_{\rm bulk}
    \simeq \frac{\frac{4\pi}{3}R_\Lambda^3}{\ell_{\rm cell}^3}.
\end{equation}
Similarly, the number of minimal surface ``pixels'' of area
$\ell_{\rm cell}^2$ on the horizon is
\begin{equation}
    N_{\rm surf}
    \simeq \frac{4\pi R_\Lambda^2}{\ell_{\rm cell}^2}.
\end{equation}
We define the holographic redundancy as
\begin{equation}
    \xi \equiv \frac{N_{\rm bulk}}{N_{\rm surf}}
    = \frac{R_\Lambda}{3\,\ell_{\rm cell}}
    = \frac{1}{\sqrt{3}\,\ell_{\rm cell}\sqrt{\Lambda}}.
\end{equation}
Solving for $\Lambda\ell_{\rm cell}^2$ gives the purely geometric scaling
\begin{equation}
    \Lambda \ell_{\rm cell}^2
    = \frac{1}{3\xi^2}.
    \label{eq:supp_geom_relation}
\end{equation}
This shows that for fixed redundancy $\xi$ the product
$\Lambda \ell_{\rm cell}^2$ is suppressed by $1/\xi^2$. The numerical
coefficient depends on the detailed definition of ``cell'' and can be
renormalized by order-one factors.

\subsection{de Sitter entropy and microscopic cell encoding}

The de Sitter horizon carries a Bekenstein-Hawking entropy
\begin{equation}
    S_{\rm dS}
    = \frac{A}{4\ell_P^2}
    = \frac{3\pi}{\Lambda\ell_P^2},
\end{equation}
where $\ell_P$ is the Planck length. In the QCA description, the total
number of microscopic states in a horizon volume is
$(\dim\mathcal{H}_{\rm cell})^{N_{\rm bulk}}$, where $\mathcal{H}_{\rm cell}$
is the local Hilbert space of a single QCA cell. However, due to
holographic redundancy, not all bulk states correspond to distinct
horizon states. If every $\xi$ bulk cells effectively encode one
independent horizon degree of freedom, the number of effective
degrees of freedom is $N_{\rm eff}=N_{\rm bulk}/\xi$ and the number of
distinguishable macrostates is
\begin{equation}
    \mathcal{N}_{\rm macro}
    \sim (\dim\mathcal{H}_{\rm cell})^{N_{\rm eff}}.
\end{equation}
Equating $\log\mathcal{N}_{\rm macro}$ with $S_{\rm dS}$ gives
\begin{equation}
    \frac{3\pi}{\Lambda\ell_P^2}
    \simeq
    \frac{N_{\rm bulk}}{\xi}
    \log\dim\mathcal{H}_{\rm cell}.
\end{equation}
Using $N_{\rm bulk}=(4\pi/3)R_\Lambda^3/\ell_{\rm cell}^3$ and
$R_\Lambda=\sqrt{3/\Lambda}$, a short algebraic manipulation yields
\begin{equation}
    \Lambda
    = \frac{16\,\ell_P^4}{3\,\ell_{\rm cell}^6}
      \frac{\log^2\!\dim\mathcal{H}_{\rm cell}}{\xi^2},
\end{equation}
or equivalently
\begin{equation}
    \Lambda\ell_{\rm cell}^2
    = \frac{16}{3}
      \left(\frac{\ell_P}{\ell_{\rm cell}}\right)^4
      \frac{\log^2\!\dim\mathcal{H}_{\rm cell}}{\xi^2}.
    \label{eq:supp_Lambda_xi_general}
\end{equation}
For a Planckian lattice $\ell_{\rm cell}\sim\ell_P$ this relation reduces
to
\begin{equation}
    \Lambda\ell_{\rm cell}^2
    = C\,\frac{1}{\xi^2},
    \qquad
    C \equiv \frac{16}{3}\log^2\!\dim\mathcal{H}_{\rm cell}.
\end{equation}
The dimension of the local Hilbert space depends on the specific
QCA encoding (number of qubits per cell, internal gauge structure,
etc.), so $C$ is an order one number determined by microscopic
architectural choices. In the concrete model analysed in
Ref.~\cite{UnifiedConstraints2025}, which uses a four-qubit cell
with constrained gauge DOF, these details conspire to give
$C\simeq 9/64$, leading to the working relation used in the
main text,
\begin{equation}
    \Lambda\ell_{\rm cell}^2
    \approx \frac{9}{64}\,\frac{1}{\xi^2}.
\end{equation}
The key point for the present work is the scaling
$\Lambda\ell_{\rm cell}^2\propto 1/\xi^2$: the enormous value of the
holographic redundancy parameter $\xi$ explains the smallness of
$\Lambda$ in terms of information redundancy rather than fine tuning.

\section{Continuous GW spectrum and QCA lattice window}
\label{sec:gw_spectrum}

\subsection{Continuous broken power-law template}

Cosmological first-order phase transitions typically produce a GW
spectrum that can be approximated by a broken power law,
\begin{equation}
    \Omega_{\rm cont}(f)
    = \Omega_*\,S(f),
\end{equation}
with a spectral shape
\begin{equation}
    S(f) =
    \begin{cases}
        \left(\dfrac{f}{f_*}\right)^{n_1}, &
        f < f_*,\\[6pt]
        \left(\dfrac{f}{f_*}\right)^{n_2}, &
        f > f_*,
    \end{cases}
\end{equation}
smoothed in practice around $f\sim f_*$.
For a sound-wave dominated transition one expects $n_1\simeq 3$ at low
frequencies, reflecting causal growth $\Omega_{\rm GW}\propto f^3$, and
a negative high-frequency slope $n_2\sim -4$ to $-5$ depending on the
details of the source.

The peak frequency $f_*$ and amplitude $\Omega_*$ can be expressed
in terms of the usual transition parameters, the strength
$\alpha=\Delta\rho/\rho_{\rm rad}$, the inverse time scale
$\beta/H_*$ and the bubble wall velocity $v_w$, as
\begin{align}
    f_*^{(0)} &\sim
    \frac{\beta}{H_*}
    \frac{a_*}{a_0}\,
    \frac{c}{R_*},
    \qquad
    R_* \sim n_*\,\ell_{\rm cell},
    \label{eq:supp_fpeak_general}\\[4pt]
    \Omega_* &\sim
    \left(\frac{H_*}{\beta}\right)
    \left(\frac{\kappa\,\alpha}{1+\alpha}\right)^2,
\end{align}
where $R_*$ is the typical bubble size at collision in the broken phase,
$a_*/a_0$ is the redshift factor from the time of GW generation to
today, and $\kappa$ is an efficiency factor for converting vacuum
energy into bulk motion. We take $R_*\sim n_*\ell_{\rm cell}$ with
$n_*\sim10^2$ as a phenomenological parametrization. A more detailed
expression using the sound-shell model can be substituted without
changing the QCA modifications discussed below.

\subsection{Lattice dispersion and cutoff frequency}

On a cubic QCA the linearized tensor perturbations obey a discrete
wave equation. In one dimension this can be written as
\begin{equation}
    \ddot{h}_n(t)
    = \frac{c^2}{\ell_{\rm cell}^2}
      \Big(h_{n+1}(t)-2h_n(t)+h_{n-1}(t)\Big)
      + \Pi_n(t),
\end{equation}
with a source term $\Pi_n(t)$ from the anisotropic stress.
Setting $\Pi_n=0$ and taking a plane-wave ansatz
$h_n(t)\propto e^{i(kn\ell_{\rm cell}-\omega t)}$ yields the lattice
dispersion relation
\begin{equation}
    \omega^2(k)
    = \frac{4c^2}{\ell_{\rm cell}^2}
      \sin^2\!\left(\frac{k\ell_{\rm cell}}{2}\right),
    \qquad
    k\in\big[-\tfrac{\pi}{\ell_{\rm cell}},\tfrac{\pi}{\ell_{\rm cell}}\big].
\end{equation}
For $k\ell_{\rm cell}\ll1$ this reduces to the continuum result
$\omega\simeq ck$, but the Brillouin zone imposes a strict ultraviolet
cutoff
\begin{equation}
    \omega_{\max} = \frac{2c}{\ell_{\rm cell}},
    \qquad
    f_{\rm cut}^{(\rm gen)}
    = \frac{\omega_{\max}}{2\pi}
    \simeq \frac{c}{\pi\ell_{\rm cell}},
\end{equation}
at the time of generation. Today the cutoff frequency is redshifted to
\begin{equation}
    f_{\rm cut}^{(0)}
    = \frac{a_*}{a_0}
      f_{\rm cut}^{(\rm gen)}
    \simeq \frac{a_*}{a_0}
           \frac{c}{\pi\ell_{\rm cell}}.
\end{equation}
Combining this with the holographic relation
$\Lambda\ell_{\rm cell}^2\propto 1/\xi^2$ shows that
$f_{\rm cut}^{(0)}\propto \xi\sqrt{\Lambda}$: once $\Lambda$ and the
redundancy parameter $\xi$ are fixed, the cutoff frequency is fixed.

\subsection{QCA-modified spectrum and example}

The QCA lattice modifies the continuous spectrum in two ways:

(i) It eliminates propagating GW modes with $|k|>\pi/\ell_{\rm cell}$,
corresponding to frequencies $f>f_{\rm cut}^{(\rm gen)}$.

(ii) It introduces interference effects near the Brillouin zone boundary,
leading to oscillatory ``Bragg'' features in the spectrum.

We model these combined effects through a universal lattice window
function $W(f)$ multiplying the continuum template,
\begin{equation}
    \Omega_{\rm QCA}(f)
    = \Omega_{\rm cont}(f)\,W(f),
\end{equation}
with
\begin{equation}
    W(f)
    =
    \left[
        \frac{\sin\big(\pi f/f_{\rm cut}^{(0)}\big)}
             {\pi f/f_{\rm cut}^{(0)}}
    \right]^2.
\end{equation}
For $f\ll f_{\rm cut}^{(0)}$ we have $W(f)\to1$, so the lattice has
no effect on long wavelength modes and the spectrum reduces to the
standard continuum result. For $f\lesssim f_{\rm cut}^{(0)}$ the window
induces a rapid suppression and oscillatory structure characteristic
of waves on a lattice, and for $f\gg f_{\rm cut}^{(0)}$ it produces a
strong power-law suppression
$W(f)\sim(\pi f/f_{\rm cut}^{(0)})^{-2}$ on top of the already decaying
continuum spectrum.

As a concrete example, consider a transition with
$\beta/H_*=100$, $\alpha=0.1$, wall velocity $v_w=0.7$, and
$n_*=100$, and take $\xi=10^{60}$ so that Eq.~(\ref{eq:unified}) of
the main text gives $\ell_{\rm cell}\sim\mathcal{O}(10^{-35}\,{\rm m})$.
Choosing $a_*/a_0$ such that $f_{\rm cut}^{(0)}\simeq 100\,{\rm Hz}$
places the lattice cutoff in the most sensitive band of the Einstein
Telescope. For typical parameters the continuum template with
$n_1=3$, $n_2=-4$ gives a peak around $f_*\sim10$--$30\,{\rm Hz}$ and
a peak amplitude $\Omega_*\sim10^{-9}$, while the QCA window $W(f)$
sharpens the high-frequency tail and produces a characteristic knee
and oscillatory suppression above $f_{\rm cut}^{(0)}$, as illustrated
in Fig.~1 of the main text.

While the precise values of $\alpha$, $\beta/H_*$ and $a_*/a_0$ depend
on the microphysics of the QCA phase transition, any realization of
the discrete lattice necessarily exhibits a hard cutoff at
$f_{\rm cut}^{(0)}$ controlled by $\ell_{\rm cell}$, and therefore by
$\Lambda$ and $\xi$. A detection of such a knee and Bragg-like
oscillations in the stochastic GW spectrum would provide strong
evidence for microscopic spacetime discreteness in the QCA framework.

\begin{thebibliography}{99}
\bibitem{PRL_Letter} H. Ma and W. Zhang, "Cosmological Phase Transitions and Gravitational Wave Signatures from Topological Mass Generation in Quantum Cellular Automata", (Companion Letter).
\bibitem{UnifiedConstraints2025} H. Ma, "Unified Constraints in Finite-Entropy QCA", (Preprint).
\end{thebibliography}

\end{document}
